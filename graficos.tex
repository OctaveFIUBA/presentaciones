\section{Gráficos Bi-dimensonales}

\begin{frame}[fragile]{Gráficos 2D plot()}

La función \verb!plot()! se encarga de realizar gráficos en ejes cartesianos. Puede dibujar más de una curva en el mismo gráfico y permite modificar algunas propiedades de diseño como los colores y estilos de las líneas. \verb!plot()! puede tomar una cantidad de argumentos variable e interpreta a los vectores o matrices como datos de las curvas y a las cadenas de texto como las propiedades de diseño. La forma general de los argumentos de \verb!plot()! son de la forma siguiente.

\begin{alertblock}{Función plot - argumentos}
plot( x1 , y1 , 'propiedad1' , 'valor1' , 'propiedad2' , 'valor2' , x2 , y2 , 'propiedad3' , 'valor3' , ... ) 

Cada valor hace referencia a la propiedad inmediatamente anterior, y cada propiedad modifica el par de datos x , y anterior.
\end{alertblock}

\end{frame}

\begin{frame}[fragile]{Gráficos 2D plot()}

Probemos algunos ejemplos para ver que hace:

\begin{lstlisting}
> t = 0:0.1:10; % Vector fila de 101 elementos.
> x = 10 - 3.*t; % También fila de 101 elementos.

> plot( x ); % Grafica los puntos ( i , x(i) )
> plot ( t , x); % Grafica los puntos ( t(i) , x(i) )
> plot (t , x , 'color' , 'r' , 'linestyle', ':')
\end{lstlisting}

El último gráfico modificó el color de la línea a rojo y la dibujó punteada. Las propiedades que se pueden modificar para cada curva son: ``linestyle'', ``linewidth'', ``color'', ``marker'', ``markersize'', ``markeredgecolor'', y ``markerfacecolor''.

Para más info sobre las propiedades poner \verb!"help plot"! en la consola de Octave.
\end{frame}

\begin{frame}[fragile]{Gráficos 2D plot()}

Por último, y porque sin etiquetas en los ejes y título del gráfico el TP rebota...

\begin{lstlisting}
> plot ( t , x);
> title ( 'Título del gráfico');
> xlabel ( 'Magnitud eje X y [unidades]');
> ylabel ( 'Magnitud eje Y y [unidades]');
> legend( 'Descripción de la curva')
\end{lstlisting}

\end{frame}

\begin{frame}[fragile]{Graficando varias curvas juntas}
Generalmente es necesario graficar distintas curvas superpuestas para compararlas. Hay muchas formas de hacer esto.

Esta es la más elegante:
\begin{lstlisting}
> plot (t1, x1, t2, x2, t3, x3); % etc
\end{lstlisting}

Y podemos decorar cada curva de la forma:
\begin{lstlisting}
> plot (t1, x1, 'color', 'r', 'linewidth', 3, ...
        t1, x1, 'color', 'b', 'linestyle', '--', ...
        t3, x3, 'color', 'k', 'linestyle', ':' ); % etc
\end{lstlisting}

Nota: Con los tres puntos le decimos a Octave que el comando sigue en la próxima linea.
\end{frame}

\begin{frame}[fragile]{Graficando varias curvas juntas}
Otra forma, un poco más \textst{enferma} rebuscada es pasar todos los datos de las absisas en una matriz, como columas o filas según sean compatibles las dimensiones.

Por ejemplo:
\begin{lstlisting}
> t = linspace(0, 10, 100)';
> x = [t, t.^2/10, t.^3/100, t.^4/1000];
> plot (t, x);
> legend ("Cuadrática", "Cúbica", "Cuarta");
> xlabel ("t");
> ylabel ("t^n / 10^n");
\end{lstlisting}

Esto puede ser útil cuando la matriz x la generamos automágicamente.
\end{frame}

\begin{frame}[fragile]{Graficando varias curvas juntas}
La forma que se usa el 99\% de las veces es con el comando \verb$hold$.

Con \verb$hold on$ le decimos a Octave que grafique una curva sobre la otra, sin borrar la anterior.

\begin{lstlisting}
> t = linspace(0, 10, 100)';
> x = t.^2;
> td = (0:10)';
> xd = td.^2 + 2*randn(length(td), 1); % ruido
> hold on;
> plot (t, x, 'b');
> plot (td, xd, 'or', 'markersize', 7);
> legend ('Modelo', 'Mediciones');
\end{lstlisting}
\end{frame}

\begin{frame}[fragile]{Usos y costumbres al plotear}
El ``estándar de facto'' a la hora de hacer curvas es el siguiente:

\begin{lstlisting}
> close all; % cierro todos los plots
> % bla bla bla
> figure; % creo una nueva figura, abre una ventana
> hold on;
> plot (...); % realizo el primer plot
> plot (...); % realizo los siguientes
> legend (...);
> xlabel (...); ylabel (...);
> print ('-dpng' 'miplot'); % Guardo el plot
\end{lstlisting}
\end{frame}

\begin{frame}[fragile]{Ejes (semi)logarítmicos}
Si es necesario usar una escala logarítmica para alguno de los ejes (o ambos), es exactamente igual que antes, sólo que hay que usar \verb$semilogx()$, \verb$semilogy()$ o \verb$loglog()$ en lugar de \verb$plot()$.

\begin{alertblock}{Bug al usar ejes logarítmicos y hold}
En caso de querer usar \verb$hold on$ junto con ejes logarítmicos (o semi), primero es necesario plotear la primer curva y \emph{después} usar el comando \verb$hold on$. Caso contrario, \verb$hold on$ crea ejes lineales.
\end{alertblock}

Es decir:

\begin{lstlisting}
> figure; % creo una nueva figura, abre una ventana
> semilogx (...); % primer plot
> hold on;
> semilogx (...); % realizo los siguientes
\end{lstlisting}
\end{frame}

\begin{frame}[fragile]{Ejes dobles independientes}
Si queremos graficar dos curvas con ejes independientes, podemos usar el comando \verb$plotyy()$. La sintaxis es:

\begin{lstlisting}
plotyy (x1, y1, x2, y2);
plotyy (x1, y1, x2, y2, @TipoPlot1, @TipoPlot2);
\end{lstlisting}

En TipoPlot debemos poner alguna de las funciones de ploteo que ya conocimos (u otras). Esto es un "puntero a función". Ya lo veremos luego ;) pero básicamente le decimos a Octave con qué función debe graficar cada curva con sus respectivos ejes.
\end{frame}

\begin{frame}[fragile]{Ejes dobles independientes}
Este ejemplo de la documentación de GNU Octave aclara los tantos:

\begin{lstlisting}
> x = 0:0.1:2*pi;
> y1 = sin (x);
> y2 = exp (x - 1);
> ax = plotyy (x, y1, x - 1, y2, @plot, @semilogy);
> xlabel ("X");
> ylabel (ax(1), "Axis 1"); % Especifico el eje a rotular
> ylabel (ax(2), "Axis 2");
\end{lstlisting}
\end{frame}

\begin{frame}[fragile]{Más funciones de ploteo}
Hay otras funciones, por ejemplo, para hacer histogramas (hist), gráficos discretos (stem), paretos (?), graficos polares, etc.

Para más info, escribir: \verb$doc plot$ en consola.
\end{frame}



\section{Gráficos Tridimensonales}
\begin{frame}[fragile]{Graficos en 3D}
Em... no me emociona escribir sobre esto ¿quién usa estas cosas? :P

Ya con lo que vimos, debe ser trivial leer lo de \verb$doc plot3$.

Si no, este enlace: \url{https://www.gnu.org/software/octave/doc/interpreter/Three_002dDimensional-Plots.html}
\end{frame}

\section{Tarea integradora}
\begin{frame}[fragile]{Tarea propuesta}
Supongamos que queremos "emular"\footnote{Eufemismo para dibujar.} un TP de Físca I. Tenemos un sistema físico, real (un péndulo, un resorte, etc), sobre el cual realizamos mediciones (con cierta incerteza) y un modelo matemático (como la ley de Hooke).

En base a las mediciones obtenidas podemos caracterizar el sistema que medimos (en español: obtener el valor de k del resorte).
\end{frame}

\begin{frame}[fragile]{Tarea propuesta}
Proponemos:

\begin{itemize}
\item Elegir un sistema físico (resorte, péndulo, resonador cuántico, etc).
\item Dado algún modelo matemático, trazar curvas de alguna magnitud medible (estiramiento vs fuerza, por ejemplo).
\item Simular que realizamos mediciones para levantar dicha curva. Esto es: Dados puntos determinados en las absisas, obtener las ordenadas. Para simular la incerteza de medición, añadir ruido gaussiano con \verb$randn()$ o uniforme con \verb$rand()$.
\item Ajustar por mínimos cuadrados para obtener los parámetros del sistema (ejemplo, la k) dadas las mediciones con incerteza. Usar \verb$fit()$ \textst{(o hacer lo de la pseudoinversa de Moore-Penrose)}.
\item Comparar la curva ideal con la curva ajustada.
\end{itemize}
\end{frame}