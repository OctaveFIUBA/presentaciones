\section{Octave como calculadora}

%%%% Calculos simples

\begin{frame}{Calculadora simple}
	Octave puede usarse como calculadora. Simplemente podemos escribir operaciones matemáticas y ver la respuesta.
\end{frame}

\begin{frame}{Calculadora simple}
  Octave puede usarse como calculadora. Simplemente podemos escribir operaciones matemáticas y ver la respuesta.
{
\metroset{block=fill}
  \begin{alertblock}{Notación}
    Las lineas que comienzan con  denotan los comandos que \emph{nosotros} escribimos en el programa, las otras son la respuesta que devuelve.
    
    Lo que sigue a un símbolo \% es un comentario.
  \end{alertblock}
}
\end{frame}

\begin{frame}[fragile]{Probemos esto} % Si no pongo [fragile] explota todo con listings
\begin{lstlisting}
> 3+2
5
> 3*5+1.1  % El separador decimal es el .
11.1
> sqrt(16) % sqrt es abreviatura de SQuare RooT
4
> i^2
-1
> (1+3i)*(2-3i/2)
> 1/2i
\end{lstlisting}
\end{frame}

%%%% Complejos
\begin{frame}[fragile]{Numeros complejos}
  Octave maneja cómodamente números complejos. La unidad imaginaria es indistintamente \verb!i! o bien \verb!j! (esta última notación se usa mucho por los electricistas y electrónicos, los matemáticos parecen preferir la primera).

  \begin{alertblock}{Buenas prácticas}
    Por desgracia, \verb!i! y \verb!j! son comúnmente usadas como nombres de variables. Para evitar problemas, siempre ponga un número adelante de la unidad imaginaria.
    Es decir, en lugar de:
    \begin{verbatim}    > 1+i\end{verbatim}
    
    Escriba:
    \begin{verbatim}    > 1+1i\end{verbatim}
  \end{alertblock}

\end{frame}
\begin{frame}[fragile]{Forma Binómica y polar} % Si no pongo [fragile] explota todo con listings
Para obtener la parte real e imaginaria, y el módulo y ángulo de un complejo, tenemos estas funciones:
\begin{lstlisting}
> real(1-2i)
1
> imag(1-2i)
-2
> abs(1-2i)
> angle(1-2i)
> conj(1-2i)
1+2i
\end{lstlisting}
\end{frame}


%%%% VARIABLES
\begin{frame}[fragile]{Usando variables}
  Una necesidad básica es guardar la información para volver a usarla después, como las memorias de las calculadoras. En Octave podemos hacer eso usando \emph{variables}.

  \begin{alertblock}{Variables}
    Para \emph{asignar} un valor a una variable, use el operador \verb!=!
    
    Ejemplo: \verb!a=3! define una variable a y le asigna el valor numérico tres.
  \end{alertblock}

\end{frame}
\begin{frame}[fragile]{Probemos esto} % Si no pongo [fragile] explota todo con listings
\begin{lstlisting}
> a = 3*7 + 2 - 7^2
> 3*a + 17
> 3a + 17 % Esto no funciona, falta el simbolo de por
> c = 2*a + 3
> d = 1000*a;  % El ; hace que no diga el resultado
> d  % Para mostrar el valor de una variable
> disp(d)  % Esto hace lo mismo per es más elegante (?)
\end{lstlisting}
\end{frame}


%%%% FUNCIONES
\begin{frame}[fragile]{Funciones Básicas}
  GNU Octave posee un conjunto de funciones y operaciones matemáticas básicas ya implementadas. Pueden descargarse más desde OctaveForge\footnote{Visitar {http://octave.sourceforge.net/}}. Algunas son:
  
    \begin{columns}[T,onlytextwidth]
    \column{0.25\textwidth}
      Redondeo
      \begin{itemize}
        \item floor() \item ceil() \item round()
      \end{itemize}

    \column{0.25\textwidth}
      Trigonometría
      \begin{itemize}
        \item sin() \item sind() \item cos() \item tanh()
      \end{itemize}

    \column{0.25\textwidth}
      Complejos
      \begin{itemize}
        \item abs() \item angle() \item real() \item imag()
      \end{itemize}
    \column{0.25\textwidth}
      Logaritmos
      \begin{itemize}
        \item log() \item log10() \item log2() \item exp()
      \end{itemize}
  \end{columns}

  Para saber qué hacen use el comando help o el comando doc seguido del nombre de la función. Por ejemplo: \verb!help sind! o \verb!doc angle!. Si no, Googleé :)

\end{frame}

\begin{frame}[fragile]{Probemos esto} % Si no pongo [fragile] explota todo con listings
\begin{lstlisting}
> abs(-3)
> abs(1+1i)+100*angle(1+1i)
> exp(2-3j)
> sin(1+2j)
> exp(2)*( cos(-3) + 1i*sin(-3) )
> sin(pi)
\end{lstlisting}
	Notemos esta pequeña diferencia:
\begin{lstlisting}
> sin(pi)
> sind(180)
\end{lstlisting}
	Las funciones trigonométricas vienen por defecto en radianes; \verb!sind!, \verb!cosd! y \verb!tand! usan grados sexagesimales.

\end{frame}

\begin{frame}[fragile]{Probemos esto} % Si no pongo [fragile] explota todo con listings
	Ahora vamos a calcular el seno de 30 grados pasando primero a mano a radianes y usando luego \verb!sin()!. Vamos a usar variables para la claridad y reutilizabilidad del código.
\begin{lstlisting}
> angulo_grados = 30;
> angulo_radianes = angulo_grados * pi / 180;
> sin(angulo_radianes)
\end{lstlisting}
	Y vemos que da lo mismo que
\begin{lstlisting}
> sind(angulo_grados)
\end{lstlisting}

\end{frame}

\begin{frame}[fragile]{Sobre los nombres de variables}
	\begin{alertblock}{Use nombres descriptivos}
	Tenga en cuenta que alguien más va a querer leer su código. Haga que sea lo más claro y entendible posible; sepa que aún esa persona puede ser usted mismo en el futuro ;)
	
	Evite abreviaciones y use siempre un mismo estilo (por ejemplo, \verb!separar_con_guiones! o \verb!separarConMayusculas!). Elija español o inglés y \emph{sea consistente} en lo posible.
	
	Es preferible (generalmente) separar las ecuaciones complicadas en varias partes en lugar de tener expresiones kilométricas inentendibles
	\end{alertblock}
\end{frame}

