\section{Introducción}

\begin{frame}[fragile]{GNU Octave}
  GNU Octave es lenguaje de programación de alto nivel destinado principalmente a \emph{cálculo numérico}. Posee capacidades de \emph{resolución de sistemas lineales y no lineales}, y para la realización de \emph{simulaciones} numéricas. Proporciona extensas \emph{capacidades gráficas} para visualización y manipulación de datos.\footnote{Tomado de \url{https://www.gnu.org/software/octave/}}
  
  GNU Octave es \emph{software libre}. Es \emph{multiplataforma}: corre sobre sistemas GNU/Linux, BSD, Windows y OS X.
\end{frame}

\begin{frame}[fragile]{Interfaz gráfica}
  Octave (desde la versión 4.0 en adelante) posee una interfaz gráfica por defecto. Vamos a trabajar sobre ella.
  
  Si tenés una versión anterior de Octave (a partir de la 3.8), podés iniciar la interfaz gráfica así:
  \begin{description}
    \item[En *NIX\footnote{Linux, OS X, BSD, etc.}] Ejecutar \verb!$ octave --force-gui!
    \item[En Windows] Botón derecho en el acceso directo > Propiedades. En el campo destino,   agregar al final \verb!--force-gui!  y apretar Aceptar.
  \end{description}
\end{frame}

